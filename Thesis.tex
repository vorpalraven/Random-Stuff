\documentclass[a4paper,12pt]{article}
\usepackage{fontspec}
  \setmainfont{Linux Libertine O}
  \setsansfont{Linux Biolinum O}
\usepackage[margin=1in]{geometry}
\usepackage{polyglossia}
  \setdefaultlanguage[variant=american]{english}
\usepackage[backend=biber]{biblatex}
  \addbibresource{/home/zeid/Desktop/Thesis/Library.bib}
\usepackage[onehalfspacing]{setspace}
\usepackage{csquotes}
\usepackage{siunitx}
\usepackage[euler]{textgreek}
\usepackage{float}
\usepackage[nottoc]{tocbibind}
\usepackage[format=plain, labelfont={it}, textfont=it]{caption}
\usepackage{graphicx}
\usepackage{booktabs}
\usepackage{titlesec}
  \titleformat{\section}[hang]
  {\sffamily\bfseries\centering\Large}{\thesection}{1em}{}
  
  \titleformat{\subsection}[hang]
  {\sffamily\bfseries\large}{\thesubsection}{1em}{}
  
  \titleformat{\subsubsection}[hang]
  {\sffamily\itshape\normalsize}{\thesubsubsection}{1em}{}

%---------------------------------------------------------------------------------------------------------------

\begin{document}

\pagenumbering{roman}

\begin{titlepage} 
  \begin{center}
  
    \vspace*{1 in}
    
    \begin{figure}
       \begin{center}
         \includegraphics[scale=0.75]{/home/zeid/Desktop/Thesis/Photos/logo.png}
        %\caption{}
         \label{fig0}
       \end{center}
     \end{figure}
    
    \sffamily \Large Impact of Hybridization of Biopolymers on Characteristics of \textkappa-Carrageenan Aerogels \\
    
    \vspace*{1 in} 
    
    \large \textbf{By: Zeid Habjoka} \\
    \large \textbf{Supervisor: Dr. Mohammad Al-Naief} \\
    
    \vspace*{1 in} 
    
    \large A Thesis Presented in Partial Requirement for B.S. Pharmaceutical/Chemical Engineering \\
    
    \vspace*{1 in}
    
    \large Department of Pharmaceutical/Chemical Engineering \\
    \large German-Jordanian University \\
    \large Jordan \\
    
    \vspace*{1 in}
    
    \large \today
    
    \vfill
    
  \end{center}
\end{titlepage}

\pagebreak

\tableofcontents

\pagebreak

\listoffigures

\pagebreak

\listoftables

\vfill

\newpage
  
\pagenumbering{arabic}

\section{Abstract}

The impact of the introduction of starch and alginate upon the surface area of \textkappa-carrageenan spherical aerogels was investigated, with the surface area of a 3\% (wt/wt) \textkappa-carrageenan aerogel coming out at \SI{336.1}{m^2\cdot g^{-1}}, a significant improvement on the value found in scientific literature. The surface area of a carrageenan-starch hybrid aerogel came out at \SI{254.}{{m^2\cdot g^{-1}}}, a decrease of 24\% from the control sample, but still an improvement on the values obtained in the literature for both substances separately.
 
\pagebreak

\section{Introduction}

\subsection{Background}

One of the many challenges of modern medicine is the efficient and effective use of medicaments. Many drugs, such as class II and IV according to the Biopharmaceutical Classification System, are poorly soluble, and as such, exhibit poor bioavailability within the bloodstream. Nanoporous materials may be used as carriers for such drugs, enhancing the effects of the drugs and protecting them from degradation, hence allowing for lower dosages to be used.\supercite{ulker_emerging_2014}

\subsection{Aims \& Objectives}

The aim of this project is to examine the impact of the hybridization of various biopolymers such as alginate and starch with \textkappa-carrageenan on the resulting aerogel's surface area, which in turn will grant us an indication of the efficacy of said gel as a carrier for medical substances. The impact of the composition of the gel will be examined, with the gels being prepared utilizing supercritical fluid extraction with carbon dioxide, and then being characterized using the BET method.

\subsection{Outlines}

The project may be summarized with the following outline:

\begin{enumerate}
	\item Preparation of the hydrogels.
	\item Drying of the hydrogels to form aerogels.
	\item Assay of the resulting aerogels' surface area.
\end{enumerate}

\pagebreak

\section{Experimental}

\subsection{Sample Preparation}

\subsubsection{Carrageenan \& starch beads}

The following procedure was utilized to prepare pure carrageenan gels to be used as control subjects, in addition to the starch-carrageenan hybrid gels, and is adapted from Quignard et. al\supercite{quignard_aerogel_2008}. 3\% (wt/wt) gels were prepared as per the proportions listed in table \ref{tab1}. The carrageenan and starch were dispersed in distilled water at 80 °C for 30 minutes, after which the mixture was added drop-wise to a 0.6 M KCl solution at room temperature. Afterwards, the beads were aged for 12 hours in this solution without stirring, filtered, and then washed with cold distilled water.

\subsubsection{Carrageenan \& alginate beads}

3\% (wt/wt) solutions of carrageenan and alginate were prepared seperatly and the appropriate amounts weighed out according to table \ref{tab2}. The gels were then mixed and added dropwise to 0.6 M KCl. Afterards, they were ageed for 12 hours without stirring, filtered, and then washed with cold distilled water.

\begin{table}[H]
\centering
\caption{Proportions utilized in producing carrageenan-starch hybrid hydrogels}
\label{tab1}
\begin{tabular}{@{}lcccc@{}}
\toprule
\textbf{\#} & \textbf{Mass (Carrageenan) / g} & \textbf{Mass (Starch) / g} & \textbf{Total Mass / g} & \textbf{Ratio} \\ \midrule
\textbf{1} & 1.5 & 1.5 & 3 & 1:1 \\
\textbf{2} & 1 & 2 & 3 & 1:2 \\
\textbf{3} & 2 & 1 & 3 & 2:1 \\
\textbf{4} & 3 & 0 & 3 & - \\ \bottomrule
\end{tabular}%
\end{table}

\begin{table}[H]
\centering
\caption{Proportions utilized in producing carrageenan-alginate hybrid hydrogels}
\label{tab2}
\begin{tabular}{@{}lcccc@{}}
\toprule
\textbf{\#} & \textbf{Mass (Carrageenan) / g} & \textbf{Mass (Alginate) / g} & \textbf{Total Mass / g} & \textbf{Ratio} \\ \midrule
\textbf{1} & 1.5 & 1.5 & 3 & 1:1 \\
\textbf{2} & 1 & 2 & 3 & 1:2 \\
\textbf{3} & 2 & 1 & 3 & 2:1 \\ \bottomrule
\end{tabular}%
\end{table}



\subsection{Supercritical drying}

The resulting hydrogels were immersed in a series of successive ethanol–water baths of increasing
alcohol concentration (30, 50, 70, 90, and 100\%) for 2 hours each, with the last one being repeated twice. Afterwards, the ethanol was removed using supercritical carbon dioxide at 40 °C and 105 bar in an HPE-lab250r autoclave (Eurotechnica). 

\subsection{Characterization}

The BET surface area was obtained via nitrogen adsorption-desorption at after being outgassed \textit{in situ} at 70 °C, using a commercial Autosorb-1 instrument (Quantachrome Corp.).

\pagebreak

\section{Literature Review \& Theoretical Background}

\subsection{Theoretical Background}

\subsubsection{Aerogel Definition \& Classification}

The problems associated with the delivery of free drugs into the body include rapid breakdown \textit{in vivo} as well as unfavorable pharmacokinetics resulting from their often poor solubility, resulting in turn in poor biodistribution.\supercite{maleki_synthesis_2016} There are already several established methods by which the solubility of such compounds may be improved upon; for example, alteration of the pH to favor the dissociation of the drug into its ionized form, the use of co-solvents, or complexation with cyclodextrins or other such materials. Each of these approaches carries with it inherent risks and advantages. For example, use of organic co-solvents is made more difficult by the fact that not many water-miscible organic solvents are suitable for human consumption, and of those that are, heavy restrictions are often placed on their use.\supercite{aulton_aultons_2013}

Nanoporous materials have the advantage of often large surface areas, allowing them to carry large amounts of medicaments and increasing their solubility. Furthermore, interesting avenues are opened up for routes of administration such as the pulmonary route, since aerogels are finely divided powders which may be inhaled and thereby subsumed rapidly into the bloodstream. In addition, aerogels are being developed for use as controlled-release systems to control the delivery of drugs to the target site where they are needed in addition to protecting the drug from biochemical degradation. This increases the effect of the drug in the body, thereby not only ensuring that a lower dose of the drug could be used than what would otherwise be needed, which, in addition to improving patient comfort and compliance, increases economic efficiency.\supercite{vallet-regi_mesoporous_2007}

Aerogels are colloidal suspensions whose liquid content has been replaced with a gas. They were first synthesized in 1931 by Samuel Kistler.\supercite{ulker_emerging_2014} Aerogels have found application in numerous fields, ranging from chemical engineering to life sciences, where a great deal of research has been conducted into their use as tissue engineering substrates, as well as implants.\supercite{stergar_review_2016} 

Aerogels are often classified according to their material of origin. They are classed in the following categories\supercite{stergar_review_2016};

\begin{description}
	\item [Inorganic] Usually made from metal alkoxides such as silica and titania.
	\item [Organic] Usually made form biopolymers such as starch, cellulose, and carrageenan.
	\item [Hybrid] Made from both substances of inorganic and organic origin, in all manner of combination.

\end{description}

Inorganic aerogels, while biocompatible, are non-biodegradable. As such, this complicates
their use in \textit{in vivo} settings and limits their overall value. Organic aerogels, on the other hand, share many of the same properties as their inorganic counterparts, with the added benefit of biodegradability, making them ideal as drug delivery systems.\supercite{ulker_emerging_2014} Plus, polysaccharide precursors such as starch, alginate, and carrageenan are cheap and plentiful, and make for a economically and environmentally sustainable technology.\supercite{garcia-gonzalez_polysaccharide-based_2011} 

\subsubsection{Aerogel Production}

Depending on the material in question, there are multiple approaches to aerogel production, but they fall roughly into four distinct phases\supercite{ulker_emerging_2014}:

\begin{description}
  \item[Hydrogel formation] Hydrogels may be synthesized as either monoliths or spheroids.\supercite{stergar_review_2016} Spherical aerogels may be produced via emulsion polymerization, in which an emulsion is prepared by mixing the hydrogel base with a suitable organic phase, along with the addition of a surfactant, after which gelation is allowed take place.  The size of the spheres may be adjusted by adjusting the concentration of the surfactant being used, as demonstrated by Gonçalves et. al.\supercite{goncalves_alginate-based_2016} The structure of the aerogel may be further modified by altering the concentration of the base material upon preparation of the hydrogel, as well as by the use of different cross-linking ions.\supercite{mallepally_superabsorbent_2013}. Gel beads may also be produced by dispersing the gel mixture directly into the gelling agent.
  
\item[Aging] In order to make certain that the gelation has completed, ensure good mechanical strength of the gels and thereby help prevent shrinkage and damage, they are placed in aging solutions of water and alcohol for a period of time.
  
\item[Solvent exchange] The hydrogels are placed in solutions of alcohol and water to dehydrate the gels, transforming them into alcogels. This process is often performed stepwise utilizing successive immersions in solutions of increasing strength to ensure that shrinkage and collapse of the pore network of the gels do not occur. 

\item[Drying] Usually done using supercritical carbon dioxide, which due to its lack of surface tension ensuring that it can pass through the gel’s matrix and dissolve the alcohol without collapsing the pore network, hence minimizing any shrinkage and loss of surface area. Freeze drying may also be utilized.

\end{description}

\subsubsection{Aerogel Characterization}

There are numerous properties that may be analyzed in order to characterize the aerogels. One such property is the BET surface area. \supercite{campbell_synthesis_1992} There are several issues, however, that complicate the process of determining this value. Firstly, depending on the mechanical strength of the material of construction, the gel's pore network may be damaged by the condensation of nitrogen within the pores. Furthermore, depending on the conditions employed during the drying stage, the gel's pores should be relatively unchanged from it's hydrogel state. However, this is difficult to determine without the the use of techniques such as thermoporometry which could alter the properties of the gel, or the use of expensive and tedious analysis techniques such as transmission electron microscopy (TEM). Other properties which prove useful for characterizing are the pore volume and size distribution. \supercite{scherer_characterization_1998} 

\pagebreak 

\subsection{Literature Review}

Carrageenans are a family of linear polysaccharides extracted primarily from red seaweed. They find a great deal of application in the food industry as gelling agents and thickeners. They are primarily divided into three distinct classes; \textkappa, \textlambda, and \textiota.\supercite{stanley_chapter_nodate} The first two form gels upon exposure to metal ions, specifically potassium and calcium, respectively. The latter does not form gels. While all three forms of carrageenan contain sulfate groups, the main difference between the groups lies in the positioning and amount of said groups.\supercite{garcia-gonzalez_polysaccharide-based_2011}

Given the well-established precedent for its use in life sciences, its is no wonder that carrageenan has seen quite a bit of research devoted to its use as a carrier. For example, Abdelghany et. al. (2017) have used \textkappa-carrageenan-stabilized chitosan/alginate nanoparticles as carriers for the anti-tuberculosis drug ethionamide. In it, 3 formulations with differing concentrations of carrageenan were evaluated as to “particles size, zeta potential, entrapment, and release”, with great potential being observed.\supercite{abdelghany_carrageenan-stabilized_2017}

Manzocco et. al. (2017) used oleogels prepared using \textkappa-carrageenan aerogels as a base, done by preparation of carrageenan aerogels of varying concentration, which in turn were converted into alcogels and then dried using supercritical carbon dioxide. Upon drying, the aerogels were converted into oleogels via oil absorption. The resulting aerogels were “... highly porous and structurally stable materials with high mechanic strength”. Furthermore, their biodegradability and high lipohilic-molecule-absorption capabilities were also remarked upon. In addition, carrageenan-based oleogels were also observed to show great promise in the life sciences.\supercite{manzocco_exploitation_2017}

Robitzer et. al. (2011) prepared several aerogels based on carbohydrate polysaccharides such as alginate, chitosan, and \textkappa-carrageenan, with the aim of examining their surface areas and porosities using nitrogen physiosorption. Electron microscopy was also used to gain a deeper understanding of the texture of the gels produced, which in turn aided in understanding the gelling mechanism and the effects of drying on the gel matrices.\supercite{robitzer_nitrogen_2011}

Ganesan \& Ratke (2014) prepared monolithic \textkappa-carrageenan aerogels using potassium thiocyanate as a gelling agent, after which the gels were dried using supercritical carbon dioxide. The aerogels made were characterized using envelope density analysis, scanning electron microscopy, nitrogen adsorption–desorption analysis, X-ray powder diffractometry, as well as IR spectroscopy. It was determined that sulfate functional groups present in \textkappa-carrageenan, as well as the ions responsible for gelation are essential in controlling the extent of shrinkage during the drying process. Furthermore, a clear correlation has been established between the concentration of \textkappa-carrageenan and the envelope density, with variation between 0.5 and 3 wt\% leading to increases in envelope density from 40 to \SI{160}{kg\cdot m^-3}.\supercite{ganesan_facile_2014}

\pagebreak

\section{Results \& Discussion}

\subsection{Results}

\begin{table}[H]
\centering
\caption{Results of BTE method surface area analysis of the aerogels}
\label{tab3}
\begin{tabular}{@{}ccc@{}}
\toprule
\textbf{Aerogel} & \textbf{Ratio} & \textbf{Surface area / m\textsuperscript{2} · g\textsuperscript{-1}} \\ \midrule
\textbf{Carrageenan} & - & 336.1 \\
\textbf{Carrageenan-starch} & 1:1 & 254.4 \\ \bottomrule
\end{tabular}
\end{table}

\begin{table}[H]
\centering
\caption{Results of BET analysis of 3\% carrageenan aerogel using nitrogen sorption at 70 °C}
\label{tab4}
\begin{tabular}{@{}cccccc@{}}
\toprule
\textbf{P/P°} & \textbf{cm\textsuperscript{3} · g\textsuperscript{-1} (STP)} & \textbf{P/P°} & \textbf{cm\textsuperscript{3} · g\textsuperscript{-1} (STP)} & \textbf{P/P°} & \textbf{cm\textsuperscript{3} · g\textsuperscript{-1} (STP)} \\ \midrule
0.050061 & 71.9228 & 0.69906 & 185.0359 & 0.64835 & 168.7913 \\
0.10258 & 82.2195 & 0.74937 & 207.2705 & 0.6 & 156.4512 \\
0.15125 & 89.317 & 0.7991 & 240.4538 & 0.54952 & 145.7574 \\
0.20135 & 95.7772 & 0.84907 & 301.6387 & 0.49976 & 136.6899 \\
0.25115 & 102.0236 & 0.89895 & 468.1798 & 0.44815 & 128.464 \\
0.3007 & 108.2022 & 0.94896 & 976.2448 & 0.40092 & 121.5017 \\
0.35012 & 114.6281 & 0.99464 & 990.226 & 0.3503 & 114.6161 \\
0.39959 & 121.363 & 0.94947 & 984.9503 & 0.29969 & 108.0397 \\
0.45274 & 129.2195 & 0.90045 & 978.7534 & 0.24896 & 101.7219 \\
0.50163 & 137.0947 & 0.85087 & 436.5959 & 0.19866 & 95.4098 \\
0.55021 & 146.0231 & 0.80076 & 271.1045 & 0.1486 & 88.9081 \\
0.59986 & 156.5678 & 0.75066 & 215.9812 & 0.096633 & 81.0361 \\
0.65012 & 169.2411 & 0.70051 & 187.839 & 0.048211 & 71.5485 \\ \bottomrule
\end{tabular}
\end{table}

\begin{table}[H]
\centering
\caption{Results of BET analysis of 3\% carrageenan -starch hybrid aerogel using nitrogen sorption at 70 °C}
\label{tab5}
\begin{tabular}{@{}cccccc@{}}
\toprule
\textbf{P/P°} & \textbf{cm\textsuperscript{3} · g\textsuperscript{-1} (STP)} & \textbf{P/P°} & \textbf{cm\textsuperscript{3} · g\textsuperscript{-1} (STP)} & \textbf{P/P°} & \textbf{cm\textsuperscript{3} · g\textsuperscript{-1} (STP)} \\ \midrule
0.051627 & 53.5041 & 0.70174 & 132.6322 & 0.64971 & 126.4546 \\
0.099646 & 60.9703 & 0.75175 & 145.4161 & 0.60038 & 117.2031 \\
0.15042 & 66.6628 & 0.7998 & 160.4707 & 0.54936 & 108.8041 \\
0.20082 & 71.5246 & 0.85324 & 188.6006 & 0.4985 & 101.2994 \\
0.24984 & 76.3514 & 0.89915 & 236.1489 & 0.4502 & 95.9577 \\
0.30067 & 81.0407 & 0.94902 & 446.5741 & 0.40097 & 90.3457 \\
0.35032 & 85.7246 & 0.99611 & 1903.7003 & 0.35024 & 85.4622 \\
0.39989 & 90.5495 & 0.94995 & 1649.5142 & 0.29973 & 80.7451 \\
0.45116 & 96.4628 & 0.90103 & 346.6088 & 0.24891 & 76.0719 \\
0.50257 & 102.1416 & 0.85018 & 211.0899 & 0.19916 & 71.2233 \\
0.55146 & 108.1334 & 0.80067 & 172.4101 & 0.14919 & 66.3886 \\
0.60113 & 114.7877 & 0.75007 & 151.0972 & 0.099418 & 60.8161 \\
0.65147 & 123.4243 & 0.70016 & 137.9095 & 0.05013 & 53.212 \\ \bottomrule
\end{tabular}
\end{table}



















\begin{figure}[H]
  \begin{center}
    \includegraphics{/home/zeid/Desktop/Thesis/Results/car01.eps}
    \caption{Adsorption isotherm of 3\% carrageenan aerogel using nitrogen sorption at 70 °C }
    \label{fig1}
  \end{center}
\end{figure}

\begin{figure}[H]
  \begin{center}
    \includegraphics{/home/zeid/Desktop/Thesis/Results/car02.eps}
    \caption{Adsorption isotherm of  carrageenan-starch hybrid aerogel using nitrogen sorption at 70 °C}
    \label{fig2}
  \end{center}
\end{figure}

\begin{figure}[H]
  \begin{center}
    \includegraphics[height=10cm, keepaspectratio]{/home/zeid/Desktop/Thesis/Photos/car_starch_before.jpg}
    \caption{Starch-carrageenan  hybrid aerogel spheres prior to drying}
    \label{fig3}
  \end{center}
\end{figure}

\begin{figure}[H]
  \begin{center}
    \includegraphics[height=10cm, keepaspectratio]{/home/zeid/Desktop/Thesis/Photos/starch_after.jpg}
    \caption{Starch-carrageenan  hybrid aerogel spheres prior to drying}
    \label{fig4}
  \end{center}
\end{figure}

\subsection{Discussion}

As may be seen in table \ref{tab3}, the surface area of the the 3\% (wt/wt) carrageenan sample used as a control came out at \SI{336.1}{m^2\cdot g^{-1}}. This is to be contrasted with the result obtained by Robitzer et. al. (2011), in which their 2.5\% \textkappa-carrageenan aerogel yielded a surface area of \SI{200}{m^2\cdot g^{-1}}, bearing in mind that their drying conditions were at apprx. 70 bar and 31.5 °C. The greater surface area may also be explained by the much longer taken to perform the solvent exchange, as the dehydration process in by Robitzer et. al. (2011) was only performed for 15 minutes per iteration.\supercite{robitzer_nitrogen_2011}

The addition of starch to the carrageenan has not contributed positively to the surface area, as may be seen in table \ref{tab3}, where the resulting surface area came out at \SI{254.}{{m^2\cdot g^{-1}}}. While still better than the value noted in the literature for both carrageenan and starch, this does not seem to be an efficient route to maximizing the surface area of the hybrid gel. It is possible, though, that different proportions of starch may yield a different result.

\pagebreak

\section{Conclusions \& Recommendations}

There are few conclusions to be drawn from this work due to a dearth of results, however, a few conclusions may still be drawn from what has been obtained.

Firstly, the  \SI{336.1}{m^2\cdot g^{-1}} surface area result obtained from the control sample shows great promise, being almost double that of the \SI{200}{m^2\cdot g^{-1}} result derived from the literature. This indicates that \textkappa-carrageenan, in fact, does not require high pressures to produce a large surface area. Furthermore, it has been shown that the addition of starch in equal proprtions to carrageenan negatively impacts the surface area of the resulting hybrid gel, causing a roughly 24\% decrease therein.

There are numerous means to extend and improve upon the scope of this project; firstly, by investigating the impact of varying proportions of starch and alginate upon the surface area, pore size and volume of the aerogels. Furthermore, the impact of other organic biopolymers such as konjac and chitosan may also be investigated. In addition, the optimization of the drying process is also another curious avenue for study, as well as the impact of the temperature profile of the hydrogel making process on the aforementioned factors. Finally, the produced aerogels should also be tested with simple drugs such as paracetamol in order to determine their loading capacities in \textit{in vitro} conditions.

\pagebreak

\section{Bibliography}

\printbibliography[heading=none]

\end{document}
